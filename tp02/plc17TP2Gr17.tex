%
% Layout retirado de http://www.di.uminho.pt/~prh/curplc09.html#notas
%
\documentclass{report}
\usepackage[portuges]{babel}
\usepackage[utf8]{inputenc}

\usepackage{url}
\usepackage{enumerate}
\usepackage{tasks}
\usepackage{graphicx}           %package para utilizar imagens em latex
\graphicspath{ {images/} }      %folder onde tem as imagens

%\usepackage{alltt}
%\usepackage{fancyvrb}
\usepackage{listings}
%LISTING - GENERAL
\lstset{
    language=c,                          % choose the language of the code
    basicstyle=\small,
    numbers=left,
    numberstyle=\tiny,
    numbersep=5pt,
    breaklines=true,
    frame=tB,
    mathescape=true,
    escapeinside={(*@}{@*)},
    }

%\lstset{ %
%   language=Flex,                          % choose the language of the code
%   basicstyle=\ttfamily\footnotesize,      % the size of the fonts that are used for the code
%   keywordstyle=\bfseries,                 % set the keyword style
%   %numbers=left,                          % where to put the line-numbers
%   numberstyle=\scriptsize,                % the size of the fonts that are used for the line-numbers
%   stepnumber=2,                           % the step between two line-numbers. If it's 1 each line
%                                           % will be numbered
%   numbersep=5pt,                          % how far the line-numbers are from the code
%   backgroundcolor=\color{white},          % choose the background color. You must add \usepackage{color}
%   showspaces=false,                       % show spaces adding particular underscores
%   showstringspaces=false,                 % underline spaces within strings
%   showtabs=false,                         % show tabs within strings adding particular underscores
%   frame=none,                             % adds a frame around the code
%   %abovecaptionskip=-.8em,
%   %belowcaptionskip=.7em,
%   tabsize=2,                              % sets default tabsize to 2 spaces
%   captionpos=b,                           % sets the caption-position to bottom
%   breaklines=true,                        % sets automatic line breaking
%   breakatwhitespace=false,                % sets if automatic breaks should only happen at whitespace
%   title=\lstname,                         % show the filename of files included with \lstinputlisting;
%                                           % also try caption instead of title
%   escapeinside={\%*}{*)},                 % if you want to add a comment within your code
%   morekeywords={*,...}                    % if you want to add more keywords to the set
%}

\usepackage{xspace}

\parindent=0pt
\parskip=2pt

\setlength{\oddsidemargin}{-1cm}
\setlength{\textwidth}{18cm}
\setlength{\headsep}{-1cm}
\setlength{\textheight}{23cm}

\def\darius{\textsf{Darius}\xspace}
\def\antlr{\texttt{AnTLR}\xspace}

\def\titulo#1{\section{#1}}
\def\super#1{{\em Supervisor: #1}\\ }
\def\area#1{{\em \'{A}rea: #1}\\[0.2cm]}
\def\resumo{\underline{Resumo}:\\ }

%\input{LPgeneralDefintions}
\title{Processamento de Linguagens e Compiladores (3º ano de Curso de LCC)\\ \textbf{Sincronizador de Legendas}\\ Relatório de Desenvolvimento}
\author{Andre Sá\\ (A76361) \and Pedro Sá\\ (A78164) \and Zacarias Pedro\\ (E7870) }
\date{\today}

\begin{document}

\maketitle

\begin{abstract}
Este relatório debruça-se sobre o estudo e desenvolvimento de filtros de texto, em particular, com o gerador de análise léxica Flex e Expressões Regulares (regexes), de forma a extrair, filtrar e transformar o conteúdo de texto.
\end{abstract}

\tableofcontents

\chapter{Introdução} \label{intro}
Legendas no formato SRT têm o seguinte formato:

\begin{verbatim}
1    1
2    00:00:48,344 --> 00:00:48,500
3    Chamada: recebida
4
5    2
6    00:00:49,707 --> 00:00:53,128
7    -Está tudo pronto?
8    -Você não tinha de me substituir.
9
10   3
11   00:00:53,328 --> 00:00:56,014
12   Eu sei, mas quero fazer um turno.
13
14   4
15   00:01:06,485 --> 00:01:08,943
16   Você gosta dele, não?
17   Gosta de observá-lo.
\end{verbatim}

As linhas 1, 5, 10, 14 contêm os identificadores de legenda. As linhas 2, 6, 11, 15 contêm os tempos de início e desaparecimento de legenda. As legendas são separadas por linha em branco.\\
Considere ainda que dispomos de legendas do mesmo filme em várias línguas, mas que frequentemente diferem no tempo inicial e duração de filme.

\begin{enumerate}[1.]
    \item Construa um sincronizador de legendas:
        \begin{verbatim}
        exe1 a.srt b.srt a1 b1 a2 b2 > b-sync.srt
        \end{verbatim}
        que recalcula os tempos de entrada e de saída das legendas de b.srt de modo que as legendas com números b1 e b2 de b fiquem com as entradas sincronizadas respectivamente com a1 e a2 de a.

    \item Construa um processador de srt que:
        \begin{itemize}
            \item retire os identificadores de legenda.
            \item coloque as legendas numa unica linha juntando-as com "|"
            \item marque com traço horizontal os intervalos com mais de 2 segundos de silêncio.
        \end{itemize}
        \begin{verbatim}
        00:00:53,328 --> 00:00:56,014 Eu sei, mas quero fazer um turno.
        00:00:56,014 --> 00:01:06,485 =================================
        00:00:53,328 --> 00:00:56,014 Você gosta dele, não?|Gosta de observá-lo.
        \end{verbatim}
\end{enumerate}

\chapter{Análise e Especificação} \label{ae}

\section{Descrição informal do problema}

\subsection{Sincronizador de Legendas}


Tendo por base o enunciado referido acima, e tendo em foco o objectivo de desenvolver algo útil no contexto do problema, decidimos que se deveria alterar o problema da primeira alínea. Propomos o seguinte:
fazer uma translação de todos os blocos de forma a que os tempos relativos das legendas do ficheiro ‘b.srt‘ se mantenham, mas a legenda com ID ‘idb‘ do ficheiro ‘b.srt‘ tenha o mesmo tempo de início da legenda com ID ‘ida‘ do ficheiro ‘a.srt‘.

\begin{verbatim}
exe1 a.srt b.srt ida idb > b-sync.srt
\end{verbatim}

\subsection{Processador de Legendas}

Esta alínea é relativamente fácil: muito resumidamente, consiste em alterar a estrutura do texto de entrada e eliminar parte da informação. Tendo uma estrutura de dados que represente completamente o formato de entrada, resolver esta alínea resume-se a criar uma função que imprima tal estrutura da forma pretendida e percorrer a estrutura aplicando esta função.

\chapter{Concepção/desenho da Resolução}

\section{Estruturas de Dados}

Para a realização deste trabalho precisamos de trabalhar com tempos e intervalos de tempo. Para tal, decidimos usar as estruturas de dados e respectivas funções da header file `time.h`. Contudo, as estruturas disponíveis não têm representação de milissegundos. Para tal, criamos uma nova estrutura, baseada na estrutura da stdlib, com milissegundos:

\lstinputlisting[firstnumber=6,linerange={6-9}]{srt.h}

Esta estrutura representa um intervalo de tempo:

\lstinputlisting[firstnumber=11,linerange={11-14}]{srt.h}

Esta estrutura representa um bloco SRT, constituindo o ID da legenda, um intervalo de tempo e o texto da legenda:

\lstinputlisting[firstnumber=16,linerange={16-20}]{srt.h}

Para guardar ficheiros completos como um conjunto ordenado de blocos SRT, decidimos usar uma biblioteca criada por um dos elementos do grupo. Esta biblioteca define um vector dinâmico e funções para o manipular.
Este vector é definido como a seguir se mostra:

\lstinputlisting[firstnumber=92,linerange={92-92,94-94,96-96,98-99}]{vec.h}

em que `VEC\_VEC` e `VEC\_DDATA\_TYPE` são macros que expandem para, respectivamente, um nome para o tipo do vector e o tipo de dados que pretendemos guardar no vector.

\section{Algoritmos}

\subsection{Processar o(s) ficheiro(s) de entrada}

Apesar do exercício pedir dois executáveis, criámos e usámos apenas um analisador léxico que lê o ficheiro de entrada na íntegra e guarda-o com o auxílio das estruturas acima mencionadas.

Desta forma, basta criar um ponto de entrada do programa diferente, a função `main`, e funções específicas para cada problema a resolver.\\

No analisador léxico existem 3 condições de contexto, `SRT\_ID`, `SRT\_INT` e `SRT\_SUB`, correspondentes, respectivamente ao ID, intervalo de tempo, e ao texto da legenda.

Nas condições `SRT\_ID` e `SRT\_INT` procuramos números com o formato esperado e, se encontrados, são lidos para a estrutura e campo certos.

Na condição `SRT\_SUB` concatenamos todo o texto até encontrarmos uma linha em branco, o que faz voltar à condição `SRT\_ID`.

No caso de ocorrerem erros, em qualquer passo do programa, mensagens de erro/diagnóstico são impressas com uma breve descrição e, se possível, a execução continua.\\

O autómato que representa o analisador léxico é o seguinte:

\begin{verbatim}
                             o ficheiro não
                             chegou  ao fim
                        +----------------------+
                        V                      |
                 --> SRT_ID --> SRT_INT --> SRT_SUB --> FIM
                            ler         ler         fim
                           um ID        um       do ficheiro
                                     intervalo
\end{verbatim}

Caso o fim do ficheiro seja encontrado fora da `SRT\_SUB`, mesmo que seja lido o ID ou o intervalo, o bloco não é adicionado ao vector e, portanto, o vector representa um ficheiro SRT completo e correcto.

\chapter{Codificação e Testes}

\section{Alternativas, Decisões e Problemas de Implementação}

\subsection{Alocação Estática de Strings}

Para a representação de um bloco SRT, o texto da legenda é uma string alocada estaticamente. Esta decisão foi tomada para simplificar a implementacão do resto do programa, embora tenha desvantagens: ao alocar espaço estaticamente, limitamos o tamanho máximo que a legenda pode ter. 
Alternativas a esta decisão seriam:

\begin{enumerate}
    \item alocar manualmente espaço, e restantes operações;
    \item criar apenas as funções necessárias, visto serem poucas.
    \item usar uma biblioteca de Strings que trata de tudo;
\end{enumerate}

A primeira opção, apesar de ser a mais simples, iria complicar muito o restante código. Achámos a nossa solução mais eficaz.\\
A segunda opção, que requer mais trabalho por nossa parte, seria simples, não complicaria o código envolvente, e seria rápida de criar.\\
A terceira opção seria a mais atraente, por requerer pouco trabalho da nossa parte e ser possível testá-la eficazmente; mas seria ainda preciso aprender a usá-la e seria uma dependência extra para eventuais utilizadores do nosso programa.

\subsection{Tempo}

Para a representação de tempo, a escolha da stdlib foi a mais óbvia. Manipular tempo não é uma tarefa simples. Mesmo assim, encontrámos limitações e tivemos de criar funções para converter entre diferentes representações e adicionar tempo, que possivelmente contêm erros.\\

\chapter{Conclusão} \label{concl}

Este trabalho foi um bom exercício de aprendizagem e práctica do `flex`, da linguagem C e da biblioteca `vec.h`. Em especial, a biblioteca foi melhorada graças ao uso da mesma neste trabalho.\\

\appendix
\chapter{Código do Programa}

Lista-se a seguir o código do analisador léxico desenvolvido.

\lstinputlisting{plc17TP2Gr17.l}

\bibliographystyle{alpha}
\bibliography{relprojLayout}
\end{document}
