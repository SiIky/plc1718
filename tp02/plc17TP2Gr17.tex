%
% Layout retirado de http://www.di.uminho.pt/~prh/curplc09.html#notas
%
\documentclass{report}
\usepackage[portuges]{babel}
\usepackage[utf8]{inputenc}
%\usepackage[latin1]{inputenc}

\usepackage{url}
\usepackage{enumerate}
\usepackage{tasks}
\usepackage{graphicx}           %package para utilizar imagens em latex
\graphicspath{ {images/} }      %folder onde tem as imagens

%\usepackage{alltt}
%\usepackage{fancyvrb}
\usepackage{listings}
%LISTING - GENERAL
\lstset{
    basicstyle=\small,
    numbers=left,
    numberstyle=\tiny,
    numbersep=5pt,
    breaklines=true,
    frame=tB,
    mathescape=true,
    escapeinside={(*@}{@*)},
    }

%\lstset{ %
%   language=Flex,                          % choose the language of the code
%   basicstyle=\ttfamily\footnotesize,      % the size of the fonts that are used for the code
%   keywordstyle=\bfseries,                 % set the keyword style
%   %numbers=left,                          % where to put the line-numbers
%   numberstyle=\scriptsize,                % the size of the fonts that are used for the line-numbers
%   stepnumber=2,                           % the step between two line-numbers. If it's 1 each line
%                                           % will be numbered
%   numbersep=5pt,                          % how far the line-numbers are from the code
%   backgroundcolor=\color{white},          % choose the background color. You must add \usepackage{color}
%   showspaces=false,                       % show spaces adding particular underscores
%   showstringspaces=false,                 % underline spaces within strings
%   showtabs=false,                         % show tabs within strings adding particular underscores
%   frame=none,                             % adds a frame around the code
%   %abovecaptionskip=-.8em,
%   %belowcaptionskip=.7em,
%   tabsize=2,                              % sets default tabsize to 2 spaces
%   captionpos=b,                           % sets the caption-position to bottom
%   breaklines=true,                        % sets automatic line breaking
%   breakatwhitespace=false,                % sets if automatic breaks should only happen at whitespace
%   title=\lstname,                         % show the filename of files included with \lstinputlisting;
%                                           % also try caption instead of title
%   escapeinside={\%*}{*)},                 % if you want to add a comment within your code
%   morekeywords={*,...}                    % if you want to add more keywords to the set
%}

\usepackage{xspace}

\parindent=0pt
\parskip=2pt

\setlength{\oddsidemargin}{-1cm}
\setlength{\textwidth}{18cm}
\setlength{\headsep}{-1cm}
\setlength{\textheight}{23cm}

\def\darius{\textsf{Darius}\xspace}
\def\antlr{\texttt{AnTLR}\xspace}
\def\pe{\emph{Filtros de Texto(GAWK)}\xspace}

\def\titulo#1{\section{#1}}
\def\super#1{{\em Supervisor: #1}\\ }
\def\area#1{{\em \'{A}rea: #1}\\[0.2cm]}
\def\resumo{\underline{Resumo}:\\ }

%%%%\input{LPgeneralDefintions}
\title{Processamento de Linguagens e Compiladores (3º ano de Curso de LCC)\\ \textbf{Sincronizador de Legendas}\\ Relatório de Desenvolvimento}
\author{Andre Sá\\ (A76361) \and Pedro Sá\\ (A78164) \and Zacarias Pedro\\ (E7870) }
\date{\today}

\begin{document}

\maketitle

\begin{abstract}
Este relatório debruça-se sobre o estudo de filtros de texto, em específico, o gerador de análise léxica Flex juntamente com outro parser, este baseado em gramática, o Yacc de forma a filtrar e/ou transformar o conteúdo de textos.
Os objectivos focam-se em reforçar o domínio da utilização de Expressões Regulares(padõres) e de gramáticas com o objectivo de extrair/filtrar/transformar o conteúdo de texto.
\end{abstract}

\tableofcontents

\chapter{Introdução} \label{intro}
O seguinte extracto de legendas formato srt tem o seguinte formato:

\begin{verbatim}
1    1
2    00:00:48,344 --> 00:00:48,500
3    Chamada: recebida
4
5    2
6    00:00:49,707 --> 00:00:53,128
7    -Está tudo pronto?
8    -Você não tinha de me substituir.
9
10   3
11   00:00:53,328 --> 00:00:56,014
12   Eu sei, mas quero fazer um turno.
13
14   4
15   00:01:06,485 --> 00:01:08,943
16   Você gosta dele, não?
17   Gosta de observá-lo.
\end{verbatim}

As linhas 1, 5, 10, 14 contêm os identificadores de legenda. As linhas 2, 6, 11, 15 contêm os tempos de início e desaparecimento de legenda.
As legendas são separadas por linha em branco. Considere ainda que dispomos lengendas do mesmo filme em várias línguas, mas que frequentemente diferem no tempo inicial e duração de filme.

\begin{enumerate}[1.]
    \item Construa um sincronizador de legendas:
        \begin{verbatim}
        exe1 a.srt b.srt a1 b1 a2 b2 > b-sync.srt
        \end{verbatim}
        que recalcula os tempos de entrada e de saída das legendas de b.srt de modo que as legendas com números b1 e b2 de b fiquem com as entradas sincronizadas respectivamente com a1 e a2 de a.

    \item Construa um processador de srt que:
        \begin{itemize}
            \item retire os identificadores de legenda.
            \item coloque as legendas numa unica linha juntando-as com "|"
            \item marque com traço horizontal os intervalos com mais de 2 segundos de silêncio.
        \end{itemize}
        \begin{verbatim}

        00:00:53,328 --> 00:00:56,014 Eu sei, mas quero fazer um turno.
        00:00:56,014 --> 00:01:06,485 =================================
        00:00:53,328 --> 00:00:56,014 Você gosta dele, não? | Gosta de observá-lo.
        \end{verbatim}
\end{enumerate}

\chapter{Análise e Especificação} \label{ae}

\section{Descrição informal do problema}
Tendo por base o enunciado referido acima e tendo em foco o objectivo de desenvolver algo útil no contexto do problem. Em grupo, decidiu-se que se deveria mudar o problema da primeira alinea.
Então em vez de só recalcular o tempo de uma legenda por base de outra, decidiu-se recalcular o tempo de todas as legendas do segundo documento, tempo que é obtido através da diferença de tempo entre uma legenda do primeiro ficheiro(o utilizador escolhe o id da legenda) e outra legenda do segundo ficheiro.

\begin{verbatim}
exe1 a.srt b.srt a1 b1 > b-sync.srt
\end{verbatim}

\chapter{Concepção/desenho da Resolução}
\section{Estruturas de Dados}

Como a estrutura `struct tm` da stdlib de C não contém milisegundos, e para o nosso trabalho são precisos milisegundos, criamos uma nova estrutura com milisegundos.
\lstinputlisting[firstnumber=6,linerange={6-9}]{srt.h} %input de um ficheiro

Esta estrutura representa um intervalo de tempo.
\lstinputlisting[firstnumber=11,linerange={11-14}]{srt.h} %input de um ficheiro

Esta estrutura representa um bloco SRT, constituindo o ID da legenda, um intervalo de tempo e a legenda.
\lstinputlisting[firstnumber=16,linerange={16-20}]{srt.h} %input de um ficheiro

Esta estrutura representa um vector dinámico, para conter todos os blocos SRT de um ficheiro `.srt`.
\lstinputlisting[firstnumber=92,linerange={92-92,94-94,96-96,98-99}]{vec.h} %input de um ficheiro

\section{Algoritmos}

\subsection{Processar o(s) ficheiro(s) de entrada}

No analisador léxico existem 3 condições de contexto, correspondentes ao ID (`SRT\_ID`), intervalo de tempo (`SRT\_INT`), e a legenda (`SRT\_SUB`).\\
Nas `SRT\_ID` e `SRT\_INT` não há muito a dizer.\\
Na `SRT\_SUB` vamos concatenando o texto lido para uma string

\chapter{Codificação e Testes}

\section{Alternativas, Decisões e Problemas de Implementação}

\chapter{Conclusão} \label{concl}
Este trabalho foi um bom exercício de aprendizagem e práctica do `awk`.

Um exemplo de uma melhoria a fazer neste programa é contar o número de relações lidas a partir dos cabeçalhos, comparar com o número de campos lidos nas linhas seguintes e fazer algo com essa informação.\\
A seguir está um exemplo de um tal bloco e respectivas linhas que não encaixam na perfeição: no cabeçalho existem 2 campos além termo base, mas só existe 1 nas linhas que lhe seguem.


\appendix
\chapter{Código do Programa}

Lista-se a seguir o código do programa que foi desenvolvido.

\lstinputlisting{plc17TP2Gr17.l} %input de um ficheiro

\bibliographystyle{alpha}
\bibliography{relprojLayout}
\end{document}
