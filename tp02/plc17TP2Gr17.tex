%
% Layout retirado de http://www.di.uminho.pt/~prh/curplc09.html#notas
%
\documentclass{report}
\usepackage[portuges]{babel}
\usepackage[utf8]{inputenc}
%\usepackage[latin1]{inputenc}

\usepackage{url}
\usepackage{enumerate}
\usepackage{tasks}
\usepackage{graphicx}           %package para utilizar imagens em latex
\graphicspath{ {images/} }      %folder onde tem as imagens

%\usepackage{alltt}
%\usepackage{fancyvrb}
\usepackage{listings}
%LISTING - GENERAL
\lstset{
    basicstyle=\small,
    numbers=left,
    numberstyle=\tiny,
    numbersep=5pt,
    breaklines=true,
    frame=tB,
    mathescape=true,
    escapeinside={(*@}{@*)}
    }
%
%\lstset{ %
%   language=Flex,                          % choose the language of the code
%   basicstyle=\ttfamily\footnotesize,      % the size of the fonts that are used for the code
%   keywordstyle=\bfseries,                 % set the keyword style
%   %numbers=left,                          % where to put the line-numbers
%   numberstyle=\scriptsize,                % the size of the fonts that are used for the line-numbers
%   stepnumber=2,                           % the step between two line-numbers. If it's 1 each line
%                                           % will be numbered
%   numbersep=5pt,                          % how far the line-numbers are from the code
%   backgroundcolor=\color{white},          % choose the background color. You must add \usepackage{color}
%   showspaces=false,                       % show spaces adding particular underscores
%   showstringspaces=false,                 % underline spaces within strings
%   showtabs=false,                         % show tabs within strings adding particular underscores
%   frame=none,                             % adds a frame around the code
%   %abovecaptionskip=-.8em,
%   %belowcaptionskip=.7em,
%   tabsize=2,                              % sets default tabsize to 2 spaces
%   captionpos=b,                           % sets the caption-position to bottom
%   breaklines=true,                        % sets automatic line breaking
%   breakatwhitespace=false,                % sets if automatic breaks should only happen at whitespace
%   title=\lstname,                         % show the filename of files included with \lstinputlisting;
%                                           % also try caption instead of title
%   escapeinside={\%*}{*)},                 % if you want to add a comment within your code
%   morekeywords={*,...}                    % if you want to add more keywords to the set
%}

\usepackage{xspace}

\parindent=0pt
\parskip=2pt

\setlength{\oddsidemargin}{-1cm}
\setlength{\textwidth}{18cm}
\setlength{\headsep}{-1cm}
\setlength{\textheight}{23cm}

\def\darius{\textsf{Darius}\xspace}
\def\antlr{\texttt{AnTLR}\xspace}
\def\pe{\emph{Filtros de Texto(GAWK)}\xspace}

\def\titulo#1{\section{#1}}
\def\super#1{{\em Supervisor: #1}\\ }
\def\area#1{{\em \'{A}rea: #1}\\[0.2cm]}
\def\resumo{\underline{Resumo}:\\ }

%%%%\input{LPgeneralDefintions}
\title{Processamento de Linguagens e Compiladores (3º ano de Curso de LCC)\\ \textbf{Sincronizador de Legendas}\\ Relatório de Desenvolvimento}
\author{Andre Sá\\ (A76361) \and Pedro Sá\\ (A78164) \and Zacarias Pedro\\ (E7870) }
\date{\today}

\begin{document}

\maketitle

\begin{abstract}
Este relatório debruça-se sobre o estudo de filtros de texto, em específico, o gerador de análise léxica Flex juntamente com outro parser, este baseado em gramática, o Yacc de forma a filtrar e/ou transformar o conteúdo de textos.
Os objectivos focam-se em reforçar o domínio da utilização de Expressões Regulares(padõres) e de gramáticas com o objectivo de extrair/filtrar/transformar o conteúdo de texto.
\end{abstract}

\tableofcontents

\chapter{Introdução} \label{intro}
O seguinte extracto de legendas formato srt tem o seguinte formato:
 
 \includegraphics[scale=0.5]{srt}

As linhas 4, 8, 13, 17 contêm os identificadores de legenda. As linhas 5, 9, 14, 18 contêm os tempos de íninio e desaparecimento de legenda.
As legendas são separadas por linha em branco. Considere ainda que dispomos lengendas do mesmo filme em várias línguas, mas que frequentemente diferem no tempo inicial e duração de filme.

\begin{enumerate}[1.]
    \item Construa um sincronizador de legendas: 
        \begin{verbatim} 
        srtsync a.srt b.srt a1 b1 a2 b2 > b-sync.srt 
        \end{verbatim}
        que recalcula os tempos de entrada e de saída das legendas de b.srt de modo que as legendas com números b1 e b2 de b fiquem com as entradas sincronizadas respectivamente com a1 e a2 de a.

    \item Construa um processador de srt que: 
        \begin{itemize}  
            \item{retire os identificadores de legenda.}  
            \item{coloque as legendas numa unica linha juntando-as com "|"}
            \item{marque com traço horizontal os intervalos com mais de 2 segundos de silêncio.}  
        \end{itemize}
        \begin{verbatim}

        00:00:53,328 --> 00:00:56,014 Eu sei, mas quero fazer um turno.
        00:00:56,014 --> 00:01:06,485 =================================
        00:00:53,328 --> 00:00:56,014 Você gosta dele, não? | Gosta de observá-lo.
        \end{verbatim}
\end{enumerate}



\chapter{Análise e Especificação} \label{ae}

\section{Descrição informal do problema}
Tendo por base o enunciado referido acima e tendo em foco o objectivo de desenvolver algo útil no contexto do problem. Em grupo, decidiu-se que se deveria mudar o problema da primeira alinea. 
Então em vez de só recalcular o tempo de uma legenda por base de outra, decidiu-se recalcular o tempo de todas as legendas do segundo documento, tempo que é obtido através da diferença de tempo entre uma legenda do primeiro ficheiro(o utilizador escolhe o id da legenda) e outra legenda do segundo ficheiro.  

\begin{verbatim}
srtsync a.srt b.srt a1 b1 > b-sync.srt 
\end{verbatim}



\section{Especificação do Requisitos}

\subsection{Dados}
\subsection{Pedidos}
\subsection{Relações}


\chapter{Concepção/desenho da Resolução}
\section{Estruturas de Dados}
Como estrutura de dados para guardar a informção lida utilizamos dicionários de dicionários.
A ordem das chaves para cada dicionário é domínio, termo base, relação, alternativa.

\section{Algoritmos}
A resolução é dividida em duas partes: processamento do ficheiro de entrada e geração das páginas HTML.

\subsection{Processar o ficheiro de entrada}
Primeiro removemos comentários da linha, se existirem.\\
De seguida, se a linha for um cabeçalho, extraímos o domínio e a sintaxe a usar para o bloco que se segue.
Caso o domínio seja vazio, assumimos um domínio universal.\\
Se a linha não for um cabeçalho, lemos os campos e guardá-mo-los na estrutura acima descrita.

\subsection{Gerar as páginas HTML}
Para gerar as páginas HTML percorremos a estrutura criada pelo processamento.\\
Para as páginas de domínio, criamos uma lista com todos os termos base, em que cada termo base tem uma lista com as relações, e cada relação tem uma lista de alternativas.\\
Para as páginas de tradução, percorremos os domínios, os termos base, e as alternativas, criando uma lista de alternativas para cada termo base.\\

\chapter{Codificação e Testes}

\section{Alternativas, Decisões e Problemas de Implementação}

\subsection{Páginas de domínio}

Uma possível escolha para organizar as listas nas páginas HTML é:

\begin{verbatim}
dominio -> relacao -> termo_base -> alternativa
\end{verbatim}

Com os blocos seguintes como exemplo, retirados do ficheiro de entrada:

\begin{verbatim}
%the()(syn:has)
cozinha: electrodoméstico | banca | utensílio de cozinha | banco de cozinha | avental
%the()(syn:nt)
ferramenta: ferramenta de carpinteiro | ferramenta agrícula
\end{verbatim}

é gerada uma lista como esta:

\begin{verbatim}
* universo
    * has
        * cozinha
            * avental
            * utensílio de cozinha
            * electrodoméstico
            * banco de cozinha
            * banca
    * nt
        * ferramenta
            * ferramenta agrícula
            * ferramenta de carpinteiro
\end{verbatim}

Esta lista, mesmo sendo correcta e mostrar as relações certas entre os termos base e as alternativas, não é muito intuitiva ao ler.\\

Por esta razão decidimos estruturar as páginas HTML da seguinte forma:

\begin{verbatim}
dominio -> termo_base -> relacao -> alternativa
\end{verbatim}

Desta forma, o exemplo acima é transformado na seguinte lista:\\

\begin{verbatim}
* universo
    * cozinha
        * has
            * avental
            * utensílio de cozinha
            * electrodoméstico
            * banco de cozinha
            * banca
    * ferramenta
        * nt
            * ferramenta agrícula
            * ferramenta de carpinteiro
\end{verbatim}

para significar que tanto 'cozinha' como 'ferramenta' pertencem ao domínio 'universo', que cozinha tem ('has') 'avental' e 'banca', entre outros, e que ferramenta tem como subclasses ('nt') 'ferramenta agrícula' e 'ferramenta de carpinteiro'.\\

\subsection{Páginas de tradução}

Para as páginas de tradução a escolha é simples. Tendo em conta que queremos gerar páginas de tradução de português para inglês e alemão, basta guardar os termos base, e as alternativas que tenham relação 'EN' ou 'DE':

\begin{verbatim}
termo_base -> alternativa
\end{verbatim}

Tomando o seguinte bloco como exemplo:

\begin{verbatim}
%the(dom=>família)(syn:EN:bt)
irmão : brother : parentesco
irmã : sister : parentesco
filha : daughter : parentesco
filho : son : parentesco
pai : father : parentesco
mãe : mother : parentesco
pais : parents : parentesco
\end{verbatim}

obtemos a seguinte lista:

\begin{verbatim}
    * irmão
        * brother
    * irmã
        * sister
    * filha
        * daughter
    * filho
        * son
    * pai
        * father
    * mãe
        * mother
    * pais
        * parents
\end{verbatim}

\chapter{Conclusão} \label{concl}
Este trabalho foi um bom exercício de aprendizagem e práctica do `awk`.

Um exemplo de uma melhoria a fazer neste programa é contar o número de relações lidas a partir dos cabeçalhos, comparar com o número de campos lidos nas linhas seguintes e fazer algo com essa informação.\\
A seguir está um exemplo de um tal bloco e respectivas linhas que não encaixam na perfeição: no cabeçalho existem 2 campos além termo base, mas só existe 1 nas linhas que lhe seguem.

\begin{verbatim}
%the(dom=>família)(syn:EN:bt)
vida : life
nascimento : birth
aniversário : birthday
morte : death
\end{verbatim}

Graças ao `awk`, ignorar estes casos é simples, basta não fazer nada.

\appendix
\chapter{Código do Programa}

Lista-se a seguir o código do programa que foi desenvolvido.

%\lstinputlisting{plc17TP1Gr17.awk} %input de um ficheiro

\bibliographystyle{alpha}
\bibliography{relprojLayout}
\end{document}
